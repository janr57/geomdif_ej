% topologia-001.tex
%
% Copyright (C) 2025 José A. Navarro Ramón <janr.devel@gmail.com>

% ....................................................................
% OBSERVACIÓN: Se puede dar formato al búffer en AUCTeX con: M-x L-buff RET
% ....................................................................

\documentclass[a4paper,10pt]{article}

\usepackage{../geomdif_res.pkg}
\usepackage{../geomdif_res.defs}

% ********************************************************************
% ******* DEFINICIONES DE ESTE EJERCICIO *****************************
% ********************************************************************
% Bloques de ejercicios
\renewcommand*{\mainsubject}{Análisis Vectorial}
\renewcommand*{\parte}{EJERCICIOS DE GEOMETRÍA DIFERENCIAL}
\renewcommand*{\tipoBloque}{Demostración}
\renewcommand*{\bloque}{1}
\renewcommand*{\hoja}{1}
\renewcommand*{\ejBloque}{1}
% Fuente: examen
% \renewcommand*{\ejExamen}{1}
\renewcommand*{\fuente}{%
  Modern Differential Geometry for Physicists. 2nd ed. - [Chris J. Isham]
}

% *****************************************************************************

\begin{document}
% ############################ ENUNCIADO ######################################
% Entrada en el índice del fichero 'pdf' 'enunciado.0'
\pdfbookmark[0]{Enunciado}{enunciado}

% --------------------------------------------------------------------
% 
% --------------------------------------------------------------------
\begin{qboxshort}
  Suponga una sucesión convergente de puntos en un espacio métrico.
  Demuestre que este límite es único.
\end{qboxshort}

% ######################### RESOLUCIÓN ###############################

% ================ APARTADO a) =======================================
% Entrada en el índice del fichero 'pdf'
% \pdfbookmark[0]{Apartado a}{a}
\pdfbookmark[0]{Solución}{sol}

\begin{soluc}

  \hspace{1em} Seguiremos un procedimiento de reducción al absurdo.
  Supongamos que tenemos un espacio métrico $\left(X,d\right)$, y una sucesión
  de puntos $(x_1, x_2, \cdots)$ de este espacio, que converge.
  Supondremos que el límite al que converge no es único, sino que consta de dos
  puntos \emph{diferentes}, $x, y\in X$. Llegaremos a la conclusión de que esto
  es imposible, por lo que el límite, de existir, debe ser único.

  \bigskip
  % ---------cutebox --------------
  \begin{cutebox}[colbacktitle=gray!60!green]{\large Propiedades de la métrica $d$}
    \begin{description}
    \item[Simétrica: ]
      \begin{equation}\label{eq:simetrica}
        d(x, y) = d(y, x)
      \end{equation}
    \item[No negatividad: ]
      \begin{equation}
        d(x, y) \geq 0 \text{ y } d(x, y) = 0 \text{ si y solo si } x = y
      \end{equation}
    \item[Desigualdad triangular: ]
      \begin{equation}\label{eq:desigualdadtriangular}
        d(x,y) \leq d(x,z) + d(z,y)
      \end{equation}
    \end{description}
  \end{cutebox} %
  % ---------------------------------------------------------------------------
  
  \medskip
  % ---------cutebox --------------
  \begin{cutebox}[colbacktitle=gray!60!green]
    {\large Convergencia de sucesiones en un espacio métrico $(X,d)$}
    Posibles definiciones:
    \begin{itemize}
    \item En un espacio métrico $(X, d)$, una sucesión converge a un punto
      $x\in X$ si, para todo número real $\varepsilon > 0$, existe un entero
      positivo $n_0$ tal que, para $n > n_0$, se cumple
      \[
        d(x_n, y) < \varepsilon
      \]
      El parámetro $\varepsilon$ representa un nivel de \emph{cercanía}, e
      implica que, a partir de cierto número de términos, todos los puntos
      se encuentran tan cerca como queramos del límite.
      
    \item En un espacio métrico $(X, d)$, una sucesión converge a un punto
      $x\in X$ si, para todo número real $\varepsilon > 0$, existe un entero
      positivo $n_0$ tal que, para $n > n_0$, se cumple que estos puntos se
      encuentran en la bola abierta $B_{\varepsilon}(x)$
      \[
        x_n \in B_{\varepsilon}(x)
      \]
      donde la bola abierta de $x$ se define como
      \[
        B_{\varepsilon}(x) := \Set{x'\in X | d(x', x) < \varepsilon}
      \]
      
    \item En un espacio métrico $(X, d)$, una sucesión converge a un punto
      $x\in X$ si, para todo número real $\varepsilon > 0$, existe un entero
      positivo $n_0$ tal que, a partir de $n > n_0$, se cumple que la cola de
      la sucesión está incluida en la bola abierta $B_{\varepsilon}(x)$
      \[
        T_n \subset B_{\varepsilon}(x)
      \]
      donde la cola se define como $T_n := \set{x_k | k > n}$.
   \end{itemize}
  \end{cutebox} %
  % ---------------------------------------------------------------------------

  \medskip
  \begin{itemize}
  \item Como la sucesión converge a $x\in X$, podemos afirmar que para
    todo real $\varepsilon > 0$, existe un entero positivo $n_{01}$ de forma
    que para $n > n_{01}$
    \begin{equation}\label{eq:dx}
      d(x_n, x) < \varepsilon
    \end{equation}
  \item Según nuestra suposición, entonces la sucesión también converge a
    $y \in X$, para todo real $\varepsilon > 0$, existe un $n_{11}$ de manera
    que para $n > n_{11}$
    \begin{equation}\label{eq:dy}
      d(x_n, y) < \varepsilon
    \end{equation}    
  \end{itemize}

  Podemos utilizar el hecho de que $x$ e $y$ son dos puntos diferentes de $X$,
  para elegir un valor de $\varepsilon$. Tomamos su valor como la mitad de la
  distancia entre $x$ e $y$
  \begin{equation}\label{eq:dxy}
    \varepsilon = \dfrac{d(x,y)}{2}
  \end{equation}

  Tomamos como $n_0$ el mayor de los valores $n_{01}$ y $n_{11}$, para
  asegurarnos de que, cuando $n > \text{máx}(n_{01}, n_{11})$, los valores $x_n$
  cumplen las desigualdades \eqref{eq:dx} y \eqref{eq:dy}.
  
  Ahora aplicamos la desigualdad triangular \eqref{eq:desigualdadtriangular}
  y la simétrica \eqref{eq:simetrica}
  \[
    d(x, y) \leq d(x, x_n) + d(x_n, y) = d(x_n, x) + d(x_n, y)
  \]

  Según \eqref{eq:dx} deducimos que $d(x,y) = 2\,\varepsilon$
  \[
    2\varepsilon \leq d(x_n, x) + d(x_n, y)
  \]

  Ahora aplicamos \eqref{eq:dx} y \eqref{eq:dy}
  \[
    2\,\varepsilon < \varepsilon + \varepsilon
  \]
  
  Lo que nos lleva a una contradicción
  \[
    \varepsilon < \varepsilon
  \]

  Por reducción al absurdo se llega a que el límite, de existir, debe ser
  único.
  
\end{soluc}

\end{document}


%%% Local Variables:
%%% coding: utf-8
%%% mode: latex
%%% TeX-engine: luatex
%%% TeX-master: t
%%% End:

