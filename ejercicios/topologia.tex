% topologia.tex
%
% Copyright (C) 2025 José A. Navarro Ramón <janr.devel@gmail.com>
%
% ---------------------------------------------------------------------------
% ---------------------------------------------------------------------------
% HOJA
% ---------------------------------------------------------------------------
% ---------------------------------------------------------------------------
%\phantomsection
%\addcontentsline{toc}{subsection}{Hoja 1}
\setcounter{isubsheet}{1}

% Añade 'Contenidos' al índice pdf.
% \bookmark[level=2,dest=section]{Hoja 1}
% Nombre de enlace: 'cc1.2'
\pdfbookmark[2]{Hoja 1}{mathj01}

\begin{ejercicio}
% --------------------------------------------------------------------------
% Convergencia única de una sucesión de puntos en un espacio métrico
% --------------------------------------------------------------------------
\item
  Suponga una sucesión convergente de puntos en un espacio métrico.
  Demuestre que este límite es único.

% ...........................................................................
  \showSolved{res/topologia}{topologia-001.pdf}
% ...........................................................................
\medskip
{\color{gray}
\hrule
}

% --------------------------------------------------------------------------
% Stronger metrics than others
% --------------------------------------------------------------------------
\item
  Demuestre que si la métrica $d^{(2)}$ induce una topología más fuerte que la
  métrica $d^{(1)}$, entonces una sucesión $d^{(2)}$-convergente es
  automáticamente $d^{(1)}$-convergente.

% ...........................................................................
  \showSolved{res/topologia}{topologia-002.pdf}
% ...........................................................................
\medskip
{\color{gray}
\hrule
}


%% --------------------------------------------------------------------------
%%  Matrices de rotación
%% --------------------------------------------------------------------------
%\item
%  \begin{subejercicio}
%  \item Pruebe que la matriz de rotación $2\times 2$
%    \[
%      \begin{pNiceMatrix}
%        \overline{A}_y\\
%        \overline{A}_z
%      \end{pNiceMatrix}
%      =
%      \begin{pNiceMatrix}
%        \cos\phi & \sin\phi\\
%        -\sin\phi & \cos\phi
%      \end{pNiceMatrix}
%      \begin{pNiceMatrix}
%        A_y\\
%        A_z
%      \end{pNiceMatrix}
%    \]
%    preserva el producto escalar. Esto es, muestre que
%    \[
%    \overline{A}_y\,\, \overline{B}_y + \overline{A}_z\,\, \overline{B}_z
%    = A_y B_y + A_z B_z
%  \]
%\item ¿Qué restricciones deben cumplir los elementos $R_{ij}$ de la matriz
%  de rotación en tres dimensiones
%  \[
%    \begin{pNiceMatrix}
%      \overline{A}_x\\
%      \overline{A}_y\\
%      \overline{A}_z
%    \end{pNiceMatrix}
%    =
%    \begin{pNiceMatrix}
%      R_{xx} & R_{xy} & R_{xz}\\
%      R_{yx} & R_{yy} & R_{yz}\\
%      R_{zx} & R_{zy} & R_{zz}\\
%    \end{pNiceMatrix}
%    \begin{pNiceMatrix}
%      A_x\\
%      A_y\\
%      A_z
%    \end{pNiceMatrix}
%  \]
%  para que se conserve la longitud de $\vvv{A}$ (para todos los vectores $\vvv{A}$)?
%  \end{subejercicio}
%
%  % ...........................................................................
%  {\footnotesize \textcolor{gray}{[No resuelto]}}
%  %\showSolved{res/matematicas}{matematicas-008.pdf}
%% ...........................................................................
%%\medskip
%%{\color{gray}
%%\hrule
%%}
%
%% #########################################################################################
%% #########################################################################################
%% #########################################################################################
%
%\clearpage
%% \phantomsection
%%\addcontentsline{toc}{subsection}{Hoja 1}
%%\setcounter{isubsheet}{2}
%\stepcounter{isubsheet} 
%
%% Añade 'Contenidos' al índice pdf.
%% \bookmark[level=2,dest=section]{Hoja 1}
%% Nombre de enlace: 'cc1.2'
%\pdfbookmark[2]{Hoja 2}{mathj02}
%
%% --------------------------------------------------------------------------
%%  Matriz de rotación
%% --------------------------------------------------------------------------
%\item Encuentre la matriz de transformación $\mmm{R}$ que describa una rotación
%  de \ang{120} alrededor de un eje que pase por el origen y por el punto $(1,1,1)$.
%  La rotación debe ser antihoraria en cuando se observa el eje hacia el origen.
%
%  % ...........................................................................
%  {\footnotesize \textcolor{gray}{[No resuelto]}}
%  %\showSolved{res/matematicas}{matematicas-009.pdf}
%% ...........................................................................
%\medskip
%{\color{gray}
%\hrule
%}
%
%  
%  % --------------------------------------------------------------------------
%  % Gradiente
%  % --------------------------------------------------------------------------
%\item Suponga que $f$ es una función de dos variables, $y$ y $z$.
%  Demuestre que el gradiente
%  $\vvv{\nabla} f = (\partial f/\partial y)\xhat{y} + (\partial f/\partial z)\xhat{z}$
%  se transforma como un vector bajo rotaciones.
%  \[
%    \begin{pNiceMatrix}
%      \overline{A}_y\\
%      \overline{A}_z
%    \end{pNiceMatrix}
%    =
%    \begin{pNiceMatrix}
%      \cos\phi & \sin\phi\\
%      -\sin\phi & \cos\phi
%    \end{pNiceMatrix}
%    \begin{pNiceMatrix}
%      A_y\\
%      A_z
%    \end{pNiceMatrix}
%  \]
%  [Pista:
%  $(\partial f/\partial\overline{y})(\partial y/\partial\overline{y})
%  + (\partial f/\partial\overline{z}) (\partial z/\partial\overline{z})$,
%  y la fórmula análoga para $\partial f/\partial\overline{z}$. Sabemos que
%  $\overline{y} = y\cos\phi + z\sin\phi$ y $\overline{z} = -y\sin\phi + z\cos\phi$;
%  ``resuelva'' estas ecuaciones para $y$ y $z$ (como funciones de $\overline{y}$ y
%  $\overline{z}$), y calcule las derivadas intermedias $\partial y/\partial\overline{y}$,
%  $\partial z/\partial\overline{y}$, etc.]
%  
%  % ...........................................................................
%  {\footnotesize \textcolor{gray}{[No resuelto]}}
%  % \showSolved{res/matematicas}{matematicas-014.pdf}
%  % ...........................................................................
%  % \medskip
%  % {\color{gray}
%  % \hrule
%  % }
  
%  \medskip
%  {\color{gray}
%    \hrule
%  }
 
  
\end{ejercicio}


%%% Local Variables:
%%% coding: utf-8
%%% mode: latex
%%% TeX-engine: luatex
%%% TeX-master: "../geomdif_ej.tex"
%%% End:
